\documentclass{article}
\usepackage[utf8]{inputenc}
\usepackage[english]{babel}
\usepackage[]{amsthm} %lets us use \begin{proof}
\usepackage[]{amssymb} %gives us the character \varnothing
\usepackage[]{setspace} %provides commands to set line spacing
\usepackage[left=0.75in, right=0.75in]{geometry}
\usepackage{hyperref}
\usepackage{xcolor}
\usepackage{soul}
\usepackage{amsmath}
\usepackage{amssymb}
\usepackage{graphicx}
\usepackage{float}
\usepackage{enumitem}

\begin{document}

\begin{center}
	\LARGE{Homework 6}\\[1em]
	\large Son Nguyen\\[1em]
	%\large \today
\end{center}

\subsection*{Problem 3.5}
\begin{enumerate}[label=(\alph*)]
	\item
	      \begin{align*}
		      \langle f |xf \rangle              & = \int_{-\infty}^{\infty} f^* x f dx                                                                                                   \\\
		                                         & = \int_{-\infty}^{\infty} x f^*f dx \quad \text{(since $x$ is real: \(x = x^*\))}                                                      \\
		                                         & = \int_{-\infty}^{\infty} (xf)^* f dx                                                                                                  \\
		                                         & = \langle xf | f \rangle \Rightarrow x^\dagger = x
		      \\ \\
		      \langle f | i f \rangle            & = \int_{-\infty}^{\infty} f^* i f dx                                                                                                   \\
		                                         & = \int_{-\infty}^{\infty} i f^* f dx                                                                                                   \\
		                                         & = \int_{-\infty}^{\infty} (-if)^* f dx                                                                                                 \\
		                                         & = \langle -if | f \rangle \Rightarrow i^\dagger = -i                                                                                   \\
		      \\
		      \langle f | \frac{d}{dx} f \rangle & = \int_{-\infty}^{\infty} f^* \frac{df}{dx}  dx                                                                                        \\
		                                         & = f^* f \Big|_{-\infty}^{\infty} - \int_{-\infty}^{\infty} \frac{df^*}{dx} f dx                                                        \\
		                                         & = -\int_{-\infty}^{\infty} \frac{df^*}{dx} f dx = \langle -\frac{d}{dx} f | f \rangle \Rightarrow -\frac{d}{dx} = \frac{d}{dx}^\dagger
	      \end{align*}
	\item
	      \begin{align*}
		      \langle f | \hat{Q} \hat{R} f \rangle & = \int_{-\infty}^{\infty} f^* \hat{Q} \hat{R} f dx                                                                                                                      \\
		                                            & = \int_{-\infty}^{\infty} f^* \hat{Q} \underbrace{\left(\hat{R} f\right)}_{g} dx                                                                                        \\
		                                            & = \langle f | \hat{Q} g \rangle = \langle \hat{Q}^\dagger f | g \rangle                                                                                                 \\
		                                            & = \int_{-\infty}^{\infty} \underbrace{(\hat{Q}^\dagger f)^*}_{h} (\hat{R} f) dx                                                                                         \\
		                                            & = \langle \hat{R}^\dagger h | f \rangle = \langle \hat{R}^\dagger \hat{Q}^\dagger f | f \rangle \Rightarrow (\hat{Q} \hat{R})^\dagger = \hat{R}^\dagger \hat{Q}^\dagger \\
	      \end{align*}
	      Next, we need to prove \((\hat{Q} + \hat{R})^\dagger = \hat{Q}^\dagger + \hat{R}^\dagger\)
	      \begin{align*}
		      \langle f (\hat{Q} + \hat{R}) | f \rangle & = \int_{-\infty}^{\infty} f^* (\hat{Q} + \hat{R}) f dx                                                                                             \\
		                                                & = \int_{-\infty}^{\infty} f^* \left(\hat{Q}f + \hat{R}f\right) dx                                                                                  \\
		                                                & = \langle f | \hat{Q} f \rangle + \langle f | \hat{R} f \rangle                                                                                    \\
		                                                & = \langle \hat{Q}^\dagger f | f \rangle + \langle \hat{R}^\dagger f | f \rangle                                                                    \\
		                                                & = \int_{-\infty}^{\infty} (\hat{Q}^\dagger f)^* f + (\hat{R}^\dagger f)^* f dx                                                                     \\
		                                                & = \int_{-\infty}^{\infty} \left(\hat{Q}^\dagger f+ \hat{R}^\dagger f\right)^* f dx                                                                 \\
		                                                & = \int_{-\infty}^{\infty} \left[\left(\hat{Q}^\dagger + \hat{R}^\dagger\right)f\right]^* f dx                                                      \\
		                                                & = \langle \left(\hat{Q}^\dagger + \hat{R}^\dagger\right) f | f \rangle \Rightarrow (\hat{Q} + \hat{R})^\dagger = \hat{Q}^\dagger + \hat{R}^\dagger \\
		      \\
		      \langle f | c \hat{Q} f \rangle           & = \int_{-\infty}^{\infty} f^* c \hat{Q} f dx                                                                                                       \\
		                                                & = c \langle f | \hat{Q} f \rangle = c \langle \hat{Q}^\dagger f | f \rangle                                                                        \\
		                                                & = \int_{-\infty}^{\infty} c (\hat{Q}^\dagger f)^* f dx                                                                                             \\
		                                                & = \int_{-\infty}^{\infty} (c^* \hat{Q}^\dagger f)^* f dx                                                                                           \\
		                                                & = \langle c^* \hat{Q}^\dagger f | f \rangle \Rightarrow (c \hat{Q})^\dagger = c^* \hat{Q}^\dagger
	      \end{align*}
	\item Equation(2.48):
	      \[\hat{a}_\pm = \frac{1}{\sqrt{2 \hbar m \omega}} (\mp i \hat{p} + m \omega x)\]
	      \begin{align*}
		      \hat{a}_+^\dagger & = \left[\frac{1}{\sqrt{2 \hbar m \omega}} (-i \hat{p} + m \omega x)\right]^\dagger                                                                                                                                         \\
		                        & = \left(\frac{1}{\sqrt{2 \hbar m \omega}}\right)^* \left[(-i \hat{p})^\dagger + (m \omega x)^\dagger\right]                                                                                                                \\
		                        & = \frac{1}{\sqrt{2 \hbar m \omega}} \left[i \hat{p}^\dagger + m \omega x\right]                                                                                                                                            \\
		                        & = \frac{1}{\sqrt{2 \hbar m \omega}} \left[i \left(-i \hbar \frac{d}{dx}\right)^\dagger + m \omega x\right]                                                                                                                 \\
		                        & = \frac{1}{\sqrt{2 \hbar m \omega}} \left[i \underbrace{\left(i \hbar \left(-\frac{d}{dx}\right)\right)}_{\hat{p}} + m \omega x\right] = \frac{1}{\sqrt{2 \hbar m \omega}} \left(i \hat{p} + m \omega x\right) = \hat{a}_- \\
		                        & = \frac{1}{\sqrt{2 \hbar m \omega}} \left[\hbar \frac{d}{dx} + m \omega x\right]
	      \end{align*}
\end{enumerate}
\subsection*{Problem 3.14}
\begin{enumerate}[label=(\alph*)]
	\item Prove equation (3.64): \([\hat{A} + \hat{B}, \hat{C}] = [\hat{A}, \hat{C}] + [\hat{B}, \hat{C}]\)
	      \begin{align*}
		      [\hat{A} + \hat{B}, \hat{C}] f(x) & = [(\hat{A} + \hat{B}) \hat{C} - \hat{C} (\hat{A} + \hat{B})] f(x)                          \\
		                                        & = \hat{A} \hat{C} f(x) + \hat{B} \hat{C} f(x) - \hat{C} \hat{A} f(x) - \hat{C} \hat{B} f(x) \\
		                                        & = \hat{A} \hat{C} f(x) - \hat{C} \hat{A} f(x) + \hat{B} \hat{C} f(x) - \hat{C} \hat{B} f(x) \\
		                                        & = [\hat{A}, \hat{C}] f(x) + [\hat{B}, \hat{C}] f(x)                                         \\
		                                        & \Rightarrow [\hat{A}, \hat{C}] + [\hat{B}, \hat{C}]
	      \end{align*}
	      Prove equation (3.65): \( [\hat{A} \hat{B}, \hat{C}] = \hat{A} [\hat{B}, \hat{C}] + [\hat{A}, \hat{C}] \hat{B} \)
	      \begin{align*}
		      [\hat{A} \hat{B}, \hat{C}]f(x)                                 & = \hat{A} \hat{B} \hat{C} f(x) - \hat{C} \hat{A} \hat{B} f(x)                                                                                 \\
		      [\hat{A} [\hat{B}, \hat{C}] + [\hat{A}, \hat{C}] \hat{B} ]f(x) & = \hat{A} \hat{B} \hat{C} f(x) \underbrace{- \hat{A} \hat{C} \hat{B} f(x) + \hat{A} \hat{C} \hat{B} f(x)}_{=0} - \hat{C} \hat{A} \hat{B} f(x) \\
		                                                                     & = [\hat{A} \hat{B}, \hat{C}] f(x) \Rightarrow [\hat{A} \hat{B}, \hat{C}] = \hat{A} [\hat{B}, \hat{C}] + [\hat{A}, \hat{C}] \hat{B}
	      \end{align*}
	\item Prove \([x^n, \hat{p}] = i \hbar n x ^{n-1}\)
	      \begin{align*}
		      [x^n, \hat{p}]f(x) & = x^n \hat{p} f(x) - \hat{p} x^n f(x)                                                       \\
		                         & = x^n \left(-i \hbar \frac{d}{dx}\right) f(x) - \left(-i \hbar \frac{d}{dx}\right) x^n f(x) \\
		                         & = -x^n i \hbar \frac{d}{dx} f(x) + i \hbar \frac{d}{dx} (x^n f(x))                          \\
		                         & = - x^n i \hbar \frac{df}{dx} + i \hbar (n x^{n-1} f(x) + x^n \frac{df}{dx})                \\
		                         & = -x^n i \hbar \frac{df}{dx} + i\hbar n x^{n-1} f(x) + i\hbar x^n \frac{df}{dx}             \\
		                         & = i\hbar n x^{n-1} f(x)
	      \end{align*}
	\item
	      \begin{align*}
		      [f(x), \hat{p}]F & = f(x) \hat{p}F - \hat{p} f(x)F                                                       \\
		                       & = f(x) \left(-i \hbar \frac{d}{dx}\right)F - \left(-i \hbar \frac{d}{dx}\right) f(x)F \\
		                       & = f(x) (-i \hbar \frac{dF}{dx}) + i \hbar \frac{df}{dx}F + i \hbar f(x) \frac{dF}{dx} \\
		                       & = i\hbar \frac{df}{dx}F \Rightarrow [f(x), \hat{p}] = i\hbar \frac{df}{dx}
	      \end{align*}
	\item Equation (2.54): \(\hat{H} = \hbar \omega (\hat{a}_- \hat{a}_+ - \frac{1}{2})\)
	      \begin{align*}
		      [\hat{H}, \hat{a}_\pm] & = \hat{H} \hat{a}_\pm - \hat{a}_\pm \hat{H}                                                                                                                                          \\
		                             & =  \hbar \omega (\hat{a}_- \hat{a}_+ - \frac{1}{2}) \hat{a}_\pm - \hat{a}_\pm \hbar \omega (\hat{a}_- \hat{a}_+ - \frac{1}{2})                                                       \\
		                             & = \hbar \omega (\hat{a}_- \hat{a}_+ \hat{a}_\pm - \frac{1}{2} \hat{a}_\pm) - \hbar \omega (\hat{a}_\pm \hat{a}_- \hat{a}_+ - \hat{a}_\pm\frac{1}{2} )                                \\
		                             & = \hbar \omega (\hat{a}_- \hat{a}_+ \hat{a}_\pm - \frac{1}{2} \hat{a}_\pm - \hat{a}_\pm \hat{a}_- \hat{a}_+ + \hat{a}_\pm\frac{1}{2} )                                               \\
		                             & = \hbar \omega (\hat{a}_- \hat{a}_+ \hat{a}_\pm - \hat{a}_\pm \hat{a}_- \hat{a}_+ )                                                                                                  \\
		                             & = \hbar \omega (\hat{a}_- \hat{a}_+ \hat{a}_- - \hat{a}_- \hat{a}_- \hat{a}_+ ) \quad \text{or} \quad  \hbar \omega (\hat{a}_- \hat{a}_+ \hat{a}_+ - \hat{a}_+ \hat{a}_- \hat{a}_+ ) \\
	      \end{align*}
	      We know that \([\hat{a}_- , \hat{a}_+] = 1\) and \([\hat{a}_+ , \hat{a}_-] = -1\)

	      \begin{align*}
		      [\hat{H}, \hat{a}_-] & = \hbar \omega (\hat{a}_- \hat{a}_+ \hat{a}_- - \hat{a}_- \hat{a}_- \hat{a}_+ ) \\
		                           & = \hbar \omega \hat{a}_- [\hat{a}_+ , \hat{a}_-] = -\hbar \omega \hat{a}_-      \\
		      [\hat{H}, \hat{a}_+] & = \hbar \omega (\hat{a}_- \hat{a}_+ \hat{a}_+ - \hat{a}_+ \hat{a}_- \hat{a}_+ ) \\
		                           & = \hbar \omega [\hat{a}_- , \hat{a}_+] \hat{a}_+ = \hbar \omega \hat{a}_+       \\ \\
		                           & \Rightarrow [\hat{H}, \hat{a}_\pm] = \pm \hbar \omega \hat{a}_\pm
	      \end{align*}
\end{enumerate}
\subsection*{Problem 3.20}
The energy-time uncertainty principle is given by:
\[\Delta E \Delta t \geq \frac{\hbar}{2}\]
From equation (3.75) we have \(\Delta E = \sigma_H\)  and \(\Delta_t = \frac{\sigma_x}{\left|\frac{d\langle x \rangle}{dt}\right|}\)
\[\Rightarrow \sigma_H  \frac{\sigma_x}{\left|\frac{d\langle x \rangle}{dt}\right|} \ge \frac{\hbar}{2} \]
\begin{equation*}
	\Psi(x,0) = A[\psi_1(x)+ \psi_2(x)]
\end{equation*}
For this equation, we have:
\[A = \frac{1}{\sqrt{2}}\]
\begin{align*}
	\langle H \rangle               & = \frac{1}{2} E_1 + \frac{1}{2} E_2 = \frac{5 \pi^2 \hbar^2}{4ma^2}                                                                                                                                                                                                                                                                                                                                                                                                           \\ \\
	\langle H^2 \rangle             & = \frac{1}{2} E_1^2 + \frac{1}{2} E_2^2 = \frac{17 \hbar^4 \pi^4}{8 m^2 a^4 }                                                                                                                                                                                                                                                                                                                                                                                                 \\ \\
	\langle x \rangle               & = \frac{a}{2} - \frac{16a}{9 \pi^2 } \cos\left(\frac{3 \hbar \pi^2 }{2 m a ^2 } t\right)                                                                                                                                                                                                                                                                                                                                                                                      \\ \\
	\langle x^2 \rangle             & = \frac{a^2}{3} - \frac{5a^2}{16 \pi^2} - \frac{16a^2}{9 \pi^2} \cos \left(\frac{3 \hbar \pi^2 }{2 m a ^2 } t\right)                                                                                                                                                                                                                                                                                                                                                          \\ \\
	\frac{d \langle x \rangle }{dt} & = \frac{8 \hbar}{3ma} \sin \left(\frac{3 \hbar \pi^2 }{2 m a ^2 } t\right)                                                                                                                                                                                                                                                                                                                                                                                                    \\ \\
	\Delta E \Delta t               & = \sqrt{\left( \frac{17 \hbar^4 \pi^4}{8 m^2 a^4 } \right) - \left(\frac{5 \pi^2 \hbar^2}{4ma^2}\right)^2} \frac{\sqrt{\left(\frac{a^2}{3} - \frac{5a^2}{16 \pi^2} - \frac{16a^2}{9 \pi^2} \cos \left(\frac{3 \hbar \pi^2 }{2 m a ^2 } t\right)\right) - \left(\frac{a}{2} - \frac{16a}{9 \pi^2 } \cos\left(\frac{3 \hbar \pi^2 }{2 m a ^2 } t\right)  \right)^2}}{\left|\frac{8 \hbar}{3ma} \sin \left(\frac{3 \hbar \pi^2 }{2 m a ^2 } t\right)\right|} \ge \frac{\hbar}{2}
\end{align*}

\subsection*{Problem 3.24}

Let operator \(\hat{Q}\) has the othonormal basis \( |v_1 \rangle, |v_2 \rangle, \dots , |v_n \rangle\), the matrix representing the operator can be written as:
\begin{align*}
	\hat{Q} |v_1 \rangle & = \hat{Q}_{11} |v_1 \rangle + \hat{Q}_{21} |v_2 \rangle + \dots + \hat{Q}_{n1} |v_n \rangle \\
	\hat{Q} |v_2 \rangle & = \hat{Q}_{12} |v_1 \rangle + \hat{Q}_{22} |v_2 \rangle + \dots + \hat{Q}_{n2} |v_n \rangle \\
	                     & \vdots                                                                                      \\
	\hat{Q} |v_n \rangle & = \hat{Q}_{1n} |v_1 \rangle + \hat{Q}_{2n} |v_2 \rangle + \dots + \hat{Q}_{nn} |v_n \rangle \\
\end{align*}
or
\begin{equation*}
	\hat{Q}|v_i \rangle = \sum_{j=1}^{n} \hat{Q}_{ji} |v_j \rangle
\end{equation*}
The goal is to find the matrix element \(\hat{Q}_{ji}\), and we can do that by taking the inner product of both sides:
\begin{equation*}
	\langle v_j | \hat{Q} | v_i \rangle = \hat{Q}_{ji}
\end{equation*}
Take the complex of both sides:
\begin{equation*}
	\langle v_j | \hat{Q} | v_i \rangle^* = \hat{Q}_{ji}^* = \langle v_i | \underbrace{\hat{Q}^\dagger}_{\hat{Q}} | v_j \rangle  = \hat{Q}_{ij}
\end{equation*}
\subsection*{Problem 3.25}
Since \(|1\rangle, |2\rangle\) are orthonormal basis, we have:
\begin{align*}
	\hat{H} |1\rangle & = H_{11} |1\rangle + H_{21} |2\rangle                                                                                        \\
	                  & = \epsilon (|1 \rangle \langle 1| - |2 \rangle \langle 2| + |1\rangle \langle 2| + |2\rangle \langle 1|) |1\rangle           \\
	                  & = \epsilon (|1 \rangle + |2 \rangle) = \underbrace{\epsilon}_{H_{11}} |1 \rangle + \underbrace{\epsilon}_{H_{21}} |2 \rangle \\
	\\
	\hat{H} |2\rangle & = H_{12} |1\rangle + H_{22} |2\rangle                                                                                        \\
	                  & = \epsilon (|1 \rangle \langle 1| - |2 \rangle \langle 2| + |1\rangle \langle 2| + |2\rangle \langle 1|) |2\rangle           \\
	                  & = \epsilon (-|2\rangle + |1\rangle) = \underbrace{\epsilon}_{H_{12}} |1 \rangle + \underbrace{-\epsilon}_{H_{22}} |2 \rangle \\
\end{align*}
So the matrix representation of \(\hat{H}\) is:
\begin{equation*}
	\hat{H} = \begin{pmatrix}
		\epsilon & \epsilon  \\
		\epsilon & -\epsilon
	\end{pmatrix}
\end{equation*}
To find the eigenvalues and eigenvectors of \(\hat{H}\), we need to solve the equation:
\begin{align*}
	H |\psi\rangle                   & = E |\psi\rangle \\
	H |\psi\rangle - EI |\psi\rangle & = 0              \\
	(H - EI) |\psi\rangle            & = 0              \\
	\Rightarrow \text{det}(H - EI)   & = 0
	\Rightarrow E = \pm \epsilon \sqrt{2}
\end{align*}
For \(E_1 = \epsilon \sqrt{2}\), its eigenvector is
\(\begin{pmatrix}
	1 \\
	\sqrt{2} -1
\end{pmatrix}\) \\ \\
For \(E_2 = -\epsilon \sqrt{2}\), its eigenvector is
\(
\begin{pmatrix}
	1 \\
	-\sqrt{2} -1
\end{pmatrix}
\)
\end{document}