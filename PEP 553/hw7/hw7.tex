\documentclass{article}
\usepackage[utf8]{inputenc}
\usepackage[english]{babel}
\usepackage[]{amsthm} %lets us use \begin{proof}
\usepackage[]{amssymb} %gives us the character \varnothing
\usepackage[]{setspace} %provides commands to set line spacing
\usepackage[left=0.75in, right=0.75in]{geometry}
\usepackage{hyperref}
\usepackage{xcolor}
\usepackage{soul}
\usepackage{amsmath}
\usepackage{amssymb}
\usepackage{graphicx}
\usepackage{float}
\usepackage{enumitem}
\begin{document}

\begin{center}
	\LARGE{Homework 7}\\[1em]
	\large Son Nguyen\\[1em]
	%\large \today
\end{center}
Equation (4.15):
\[\psi(r, \theta, \phi) = R(r)Y(\theta, \phi )\]
Where \(R(r)\) is the radial wave function and \(Y(\theta, \phi)\) is the angular wave function.

\subsection*{Problem 4.3}

\begin{enumerate}[label=(\alph*)]
	\item
	      \[\psi(r, \theta, \phi) = Ae^{\frac{-r}{a}}\]
	      From equation (4.8) we have:
	      \begin{align*}
		      E \psi        & = -\frac{\hbar^2}{2m} \nabla^2 \psi + V\psi                                                                                 \\
		                    & = -\frac{\hbar^2}{2m} \frac{1}{r^2} \frac{\partial}{\partial r } \left(r^2 \frac{\partial \psi}{\partial r}\right) + V \psi \\
		                    & \text{Divide both sides by } \psi                                                                                           \\
		      \Rightarrow E & = -\frac{\hbar^2}{2m} \frac{1}{r^2 \psi} \frac{\partial}{\partial r } \left(r^2 \frac{\partial \psi}{\partial r}\right) + V \\
		      \Rightarrow V & = E + \frac{\hbar^2}{2m} \frac{1}{r^2 \psi} \frac{\partial}{\partial r } \left(r^2 \frac{\partial \psi}{\partial r}\right)  \\
	      \end{align*}
	      Where \(\nabla^2 = \frac{1}{r^2} \frac{\partial}{\partial r } \left(r^2 \frac{\partial}{\partial r}\right)\) is the Laplacian operator.\\
	      Plug in \(\psi(r, \theta, \phi) = Ae^{\frac{-r}{a}}\) into the equation above, we have:
	      \begin{align*}
		      V & = E + \frac{\hbar^2}{2m} \frac{1}{r^2 (Ae^{\frac{-r}{a}})} \frac{\partial}{\partial r } \left(r^2 \frac{\partial Ae^{\frac{-r}{a}}}{\partial r}\right) \\
		        & =  E + \frac{\hbar^2}{2 a^2 m } - \frac{\hbar^2 }{amr} = E - \frac{\hbar^2 }{2ma^2} \left(\frac{2a}{r} - 1\right)
	      \end{align*}
	      As \(r \rightarrow \infty\), \(V(r) \rightarrow 0\).
	      \begin{align*}
		      \lim_{r \to \infty} V(r) & = E - \frac{\hbar^2}{2ma^2} (-1) = 0 \\
		      \Rightarrow E            & = -\frac{\hbar^2}{2ma^2}             \\
		      \Rightarrow V(r)         & = -\frac{\hbar^2}{amr}
	      \end{align*}
	\item
	      \[\psi(r, \theta, \phi) = A e^{\frac{-r^2}{a^2}}\]
	      \begin{align*}
		      V(r) & = E + \frac{\hbar^2}{2m} \frac{1}{r^2 \psi} \frac{\partial}{\partial r } \left(r^2 \frac{\partial \psi}{\partial r}\right)                                       \\
		           & = E + \frac{\hbar^2}{2m} \frac{1}{r^2 (A e^{\frac{-r^2}{a^2}})} \frac{\partial}{\partial r } \left(r^2 \frac{\partial A e^{\frac{-r^2}{a^2}}}{\partial r}\right) \\
		           & = E - \frac{3\hbar^2}{a^2m} + \frac{2\hbar^2 r^2}{a^4 m}
	      \end{align*}
	      As \(V(0) = 0\)
	      \begin{align*}
		      \lim_{r \to \infty} V(r) & = E - \frac{3\hbar^2}{a^2m} + \frac{2\hbar^2 r^2}{a^4 m} = 0 \\
		      \Rightarrow E            & = \frac{3\hbar^2}{a^2m}                                      \\
		      \Rightarrow V(r)         & = \frac{2\hbar^2 r^2}{a^4 m}
	      \end{align*}
\end{enumerate}

\subsection*{Problem 4.4}
Equation(4.27):
\begin{equation*}
	P_l^m (x) = (-1)^m (1-x^2)^{\frac{m}{2}} \left(\frac{d}{dx}\right)^m P_l(x)
\end{equation*}
Where \(P_l(x)\) is the \(l^{th}\) Lengendre polynomial. Equation (4.28) \\
\[P_l(x) = \frac{1}{2^l l!} \left(\frac{d}{dx}\right)^l (x^2 - 1)^l\]

\noindent The normalized angular wave functions is called spherical harmonics. Equation (4.32)
\begin{equation*}
	Y_l^m(\theta, \phi) = \sqrt{\frac{(2l+1)}{4 \pi} \frac{(l - m)!}{(l + m)!}} e^{im \phi} P_l^m(\cos \theta)
\end{equation*}

\noindent Constructing \(Y_0^0, Y_2^1\)
\begin{align*}
	Y_0^0 & = \left(\frac{1}{4 \pi}\right)^{\frac{1}{2}}                                        \\
	Y_2^1 & = - \left(\frac{15}{8 \pi}\right)^{\frac{1}{2}} \sin(\theta)\cos(\theta) e^{i \phi}
\end{align*}
To check orthogonality, we need to integrate the product of the two spherical harmonics over the unit sphere.
\begin{align*}
	\int_{0}^{\pi} \int_{0}^{2\pi} Y_0^0 Y_2^1 \sin(\theta) d\theta d\phi & =  \int_{0}^{\pi} \int_{0}^{2\pi}  \left[\left(\frac{1}{4 \pi}\right)^{\frac{1}{2}}\right]^* \left[- \left(\frac{15}{8 \pi}\right)^{\frac{1}{2}} \sin(\theta)\cos(\theta) e^{i \phi}\right] \sin(\theta) d\theta d\phi \\
	                                                                      & = 0 \quad \text{(Wolfram Alpha)}                                                                                                                                                                                       \\
	                                                                      & \Rightarrow \text{Orthogonal}
\end{align*}
To normalize the spherical harmonics for \(Y_0^0, Y_2^1\), we have:
\begin{align*}
	\int_{0}^{\pi} \int_{0}^{2 \pi} |Y_0^0|^2 \sin(\theta) d\theta d\phi & = \int_{0}^{\pi} \int_{0}^{2 \pi} \left[\left(\frac{1}{4 \pi}\right)^{\frac{1}{2}} \right]^2 \sin(\theta) d\phi  d\theta = 1                                       \\
	\int_{0}^{\pi} \int_{0}^{2 \pi} |Y_2^1|^2 \sin(\theta) d\theta d\phi & = \int_{0}^{\pi} \int_{0}^{2 \pi} \left[- \left(\frac{15}{8 \pi}\right)^{\frac{1}{2}} \sin(\theta)\cos(\theta) e^{i \phi}\right]^2 \sin(\theta) d\phi  d\theta = 1
\end{align*}
*Calculated on Wolfram

\subsection*{Problem 4.7}
Find \(Y_l^l(\theta, \phi)\), and \(Y_3^2(\theta, \phi)\), we have:
\[P_3^2 = 15 \sin^2(\theta) \cos(\theta)\]
Plug equation (4.28) into equation (4.27), we have:
\begin{align*}
	P^m_l &= (-1)^m (1 - x^2)^{\frac{m}{2}} \left(\frac{d}{dx}\right)^m P_l(x) \\
	&= (-1)^m (1 - x^2)^{\frac{m}{2}} \left(\frac{d}{dx}\right)^m \frac{1}{2^l l!} \left(\frac{d}{dx}\right)^l (x^2 - 1)^l \\
	&= \frac{(-1)^m}{2^l l!} (1 - x^2)^{\frac{m}{2}} \left(\frac{d}{dx}\right)^{l+m} (x^2 - 1)^l
\end{align*}
For \(Y^2_3(\theta, \phi)\):
\begin{align*}
	Y^2_3(\theta, \phi) &= \sqrt{\frac{(2\cdot 3) + 1}{4 \pi} \frac{1!}{5!}} e^{2i \phi} P_3^2(\cos \theta) \\
	&= \sqrt{\frac{7}{4 \pi} \frac{1}{120}} e^{2i \phi}  P_3^2(\cos \theta)\\
	&= \sqrt{\frac{7}{480\pi}} e^{2i \phi}  \left[\frac{1}{2^3 \cdot 3!} (1 - x^2) \left(\frac{d}{dx}\right)^5 (x^2 - 1)^3\right]_{x = \cos \theta} \\
	&= \sqrt{\frac{7}{480\pi}} e^{2i \phi}  \left[\frac{1}{48} (1 - x^2) 720x\right]_{x = \cos \theta}\\
	&= \sqrt{\frac{7}{480\pi}} e^{2i \phi} \left[15(1-\cos^2(\theta))\cos(\theta)\right]
\end{align*}
Equation (4.18):
\begin{equation*}
	\sin(\theta) \frac{\partial}{\partial \theta} \left(\sin(\theta) \frac{\partial Y}{\partial \theta}\right) + \frac{\partial^2 Y}{\partial \phi^2} = -l(l+1) \sin^2(\theta) Y
\end{equation*}
Plug in \(Y^2_3(\theta, \phi)\) into the left hand side of the equation above, we have:
\begin{align*}
	&\sin(\theta) \frac{\partial}{\partial \theta} \left[\sin(\theta) \frac{\partial}{\partial \theta} \left(\sqrt{\frac{7}{480\pi}} e^{2i \phi} \left[15(1-\cos^2(\theta))\cos(\theta)\right]\right)\right] + \frac{\partial^2}{\partial \phi^2} \left(\sqrt{\frac{7}{480\pi}} e^{2i \phi} \left[15(1-\cos^2(\theta))\cos(\theta)\right]\right) \\
	&= -3e^{2i\phi} \sqrt{\frac{105}{2 \pi}} \cos(\theta) \sin^4(\theta) \quad \text{(Wolfram Alpha)}
\end{align*}
For the right hand side of the equation above, we have:
\begin{align*}
	&-3(3+1) \sin^2(\theta) \sqrt{\frac{7}{480\pi}} e^{2i \phi} \left[15(1-\cos^2(\theta))\cos(\theta)\right] \\
	&= -3e^{2i\phi} \sqrt{\frac{105}{2 \pi}} \cos(\theta) \sin^4(\theta) \quad \text{(Wolfram Alpha)}
\end{align*}
For \(Y^l_l (\theta, \phi)\):
\begin{align*}
	Y^l_l(\theta, \phi) &= \sqrt{\frac{2l + 1}{4 \pi} \frac{(l-l)!}{(l+l)!}} e^{il\phi} P^l_l(\cos \theta) \\
	&= \sqrt{\frac{2l + 1}{4\pi (2l)!}} e^{il\phi} \left[\frac{(-1)^l}{2^l l!} (1-x^2)^{\frac{l}{2}} \underbrace{\left(\frac{d}{dx}\right)^{2l} (x^2 - 1)^l}_{(2l)!} \right]_{x = \cos \theta} \\
	&= \frac{(-1)^l}{2^l l!} \sqrt{\frac{(2l+1)!}{4\pi}} e^{il\phi} \sin^l(\theta)
\end{align*}
Check satisfication of equation (4.18) for \(Y^l_l(\theta, \phi)\), plug in \(Y^l_l(\theta, \phi)\) into the left hand side of the equation above, we have:
\begin{align*}
	& \sin(\theta) \frac{\partial}{\partial \theta} \left(\sin(\theta) \frac{\partial}{\partial \theta} \left[\frac{(-1)^l}{2^l l!} \sqrt{\frac{(2l+1)!}{4\pi}} e^{il\phi} \sin^l(\theta)\right]\right) + \frac{\partial^2}{\partial \phi^2} \left[\frac{(-1)^l}{2^l l!} \sqrt{\frac{(2l+1)!}{4\pi}} e^{il\phi} \sin^l(\theta)\right] \\
	&= \frac{(-1)^l}{2^l l!} \sqrt{\frac{(2l+1)!}{4\pi}} \left[ \left(\sin^2(\theta) \frac{\partial^2}{\partial \theta^2} e^{il\phi} \sin^l(\theta)\right) + \left(\frac{\partial^2}{\partial \phi^2} e^{il\phi} \sin^l(\theta)\right)\right] \\
	&= \frac{(-1)^l}{2^l l!} \sqrt{\frac{(2l+1)!}{4\pi}} \left[ \left(l^2 e^{il\phi}\sin^l(\theta) \cos^2(\theta) - le^{il\phi} \sin^{(l+2)}(\theta)\right) + \left(-e^{il\phi} l^2 \sin^l(\theta)\right) \right]\\
	&= \frac{(-1)^l}{2^l l!} \sqrt{\frac{(2l+1)!}{4\pi}} (-l \sin^2(\theta) - \sin^2(\theta)) le^{il\phi} \sin^l(\theta) \\
	&= -l(l+1) \sin^2(\theta) \frac{(-1)^l}{2^l l!} \sqrt{\frac{(2l+1)!}{4\pi}} e^{il\phi} \sin^l(\theta) \\ 
	&= -l(l+1) \sin^2(\theta) Y^l_l(\theta, \phi)
\end{align*}
\subsection*{Problem 4.8}
For \(l \neq l'\)
\begin{align*}
	\int_{-1}^{1} P_l(x) P_{l'}(x) dx &= \int_{-1}^{1} \frac{1}{2^l l!} \left(\frac{d}{dx}\right)^l (x^2 - 1)^l \frac{1}{2^{l'} l'!} \left(\frac{d}{dx}\right)^{l'} (x^2 - 1)^{l'} dx \\
	&= 0
\end{align*}

For \(l = l'\)
\begin{align*}
	\int_{-1}^{1} P_l(x) P_{l'}(x) dx &= \int_{-1}^{1} \frac{1}{2^l l!} \left(\frac{d}{dx}\right)^l (x^2 - 1)^l \frac{1}{2^{l} l!} \left(\frac{d}{dx}\right)^{l} (x^2 - 1)^{l} dx \\
	&= \frac{2}{2l+1}\\ \\
	&\Rightarrow \int_{-1}^{1} P_l(x) P_{l'}(x) dx = \left(\frac{2}{2l+1}\right) \delta_{ll'}
\end{align*}

\end{document}