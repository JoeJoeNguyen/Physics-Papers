\documentclass[12pt]{article}
\usepackage[utf8]{inputenc}
\usepackage{amsmath, amssymb}
\usepackage[left=0.75in, right=0.75in]{geometry}
\usepackage{graphicx}
\usepackage[colorlinks=true, linkcolor=blue, urlcolor=red, citecolor=blue]{hyperref}
\usepackage{hyperref}
\usepackage{braket}
\usepackage{float}

\title{Optical Experiment Manuscript}
\author{Son Nguyen}
\date{\today}

\begin{document}

\maketitle
\section{Introduction}
In this manuscript, we will go through the detail of the optical experiment setup of the quantum circuit for Variational Quantum Eigensolver (VQE).
\section{Components}
\href{https://en.wikipedia.org/wiki/Jones_calculus#Phase_retarders}{Reference} \\ \\
Quarter-wave plate with fast axis at angle \(\theta\) with respect to the horizontal axis (QWP):
\begin{equation}
	e^{\frac{-i \pi}{4}}
	\begin{bmatrix}
		\cos^2 (\theta) + i \sin^2 (\theta) & (1-i) \sin(\theta)\cos(\theta)     \\
		(1-i) \sin(\theta)\cos(\theta)      & \sin^2 (\theta) + i \cos^2(\theta)
	\end{bmatrix}
\end{equation}
\begin{figure}[H]
    \centering
    \includegraphics[width=0.25\textwidth]{QWP.png}
    \caption{Quarter-wave plate}
    \label{fig:qwp}
\end{figure}
\noindent Quarter-wave plate with fast axis vertical (QWPv):
\begin{equation}
	e^{\frac{i \pi}{4}}
	\begin{bmatrix}
		1 & 0  \\
		0 & -i
	\end{bmatrix}
\end{equation}
Quarter-wave plate with fast axis horizontal (QWPh):
\begin{equation}
	e^{\frac{-i \pi}{4}}
	\begin{bmatrix}
		1 & 0 \\
		0 & i
	\end{bmatrix}
\end{equation}
Half-wave plate (HWP):
\begin{equation}
	\begin{bmatrix}
		\cos(2\phi) & \sin(2\phi)  \\
		\sin(2\phi) & -\cos(2\phi)
	\end{bmatrix}
\end{equation}
\begin{figure}[H]
    \centering
    \includegraphics[width=0.25\textwidth]{HWP.png}
    \caption{Half-wave plate}
    \label{fig:hwp}
\end{figure}
\href{https://www.thorlabs.com/newgrouppage9.cfm?objectgroup_id=146}{Dove Prism} (DP) using Rotation matrix acting on the spatial mode:
\begin{equation}
	\begin{bmatrix}
		\cos(\omega) & -\sin(\omega) \\
		\sin(\omega) & \cos(\omega)
	\end{bmatrix}
\end{equation}
\begin{figure}[H]
    \centering
    \includegraphics[width=0.3\textwidth]{DP.png}
    \caption{Dove Prism}
    \label{fig:dp}
\end{figure}

Beamsplitter (BS):
\begin{figure}[H]
    \centering
    \includegraphics[width=0.25\textwidth]{BS.png}
    \caption{Beamsplitter}
    \label{fig:bs}
\end{figure}
Polarizing beamsplitter (PBS):
\begin{samepage}
\nopagebreak[4]
\begin{figure}[H]
    \centering
    \includegraphics[width=0.35\textwidth]{PBS.png}
    \caption{Polarizing Beamsplitter}
    \label{fig:pbs}
\end{figure}
\end{samepage}
\begin{figure}[H]
	\centering
	\includegraphics[width=0.7\textwidth, height=0.3\textheight]{poincare.jpeg}
	\caption{\href{https://opg.optica.org/ol/fulltext.cfm?uri=ol-24-7-430&id=37205}{Poincare}}
\end{figure}
For this experiment, we will use "01"/"10" Spatial Mode as our second qubit and Polarization as our first qubit.
\begin{align*}
	 & \ket{0} = \text{Spatial Mode 10 and Horizontal Polarization or } \ket{0}_s, \ket{0}_p                                                                              \\
	 & \ket{1} = \text{Spatial Mode 01 and Vertical Polarization or } \ket{1}_s, \ket{1}_p                                                                                \\
	 & \frac{\ket{0}-i\ket{1}}{\sqrt{2}} = \text{South Pole of the Poincare Sphere or Left-circular Polarization} \frac{\ket{L}_s}{\sqrt{2}}, \frac{\ket{L}_p}{\sqrt{2}}  \\
	 & \frac{\ket{0}+i\ket{1}}{\sqrt{2}} = \text{North Pole of the Poincare Sphere or Right-circular Polarization} \frac{\ket{R}_s}{\sqrt{2}}, \frac{\ket{R}_p}{\sqrt{2}} \\
	 & \ket{+} = \text{Diagonal Polarization} \frac{\ket{D}_s}{\sqrt{2}}, \frac{\ket{D}_p}{\sqrt{2}}                                                                      \\
	 & \ket{-} = \text{Anti-diagonal Polarization} \frac{\ket{A}_s}{\sqrt{2}}, \frac{\ket{A}_p}{\sqrt{2}}
\end{align*}
Therefore our initial state is \( \ket{0}_p \otimes \ket{0}_s \equiv \ket{0_p 0_s}\)
\begin{figure}[H]
	\centering
	\includegraphics[width=0.7\textwidth, height=0.3\textheight]{phaseShift.jpeg}
	\caption{\href{https://opg.optica.org/ol/fulltext.cfm?uri=ol-24-7-430&id=37205}{Geometric Phase Shift}}
\end{figure}
\section{Gates Realization}
\subsection{\(R_x\) Gate}
\begin{equation}
    R_x(\theta) =
    \begin{bmatrix}
        \cos(\frac{\theta}{2}) & -i \sin(\frac{\theta}{2}) \\
        -i \sin(\frac{\theta}{2}) & \cos(\frac{\theta}{2})
    \end{bmatrix}
\end{equation}
\begin{equation}
    R_x(\frac{\pi}{2}) = 
    \begin{bmatrix}
        \frac{1}{\sqrt{2}} & -\frac{i}{\sqrt{2}} \\
        -\frac{i}{\sqrt{2}} & \frac{1}{\sqrt{2}}
    \end{bmatrix}
    = \frac{1}{\sqrt{2}}
    \begin{bmatrix}
        1 & -i \\
        -i & 1
    \end{bmatrix}
\end{equation}
\noindent \(R_x\) acting on \(\ket{0}\) gives:
\begin{equation}
    R_x(\frac{\pi}{2}) \ket{0} = \frac{1}{\sqrt{2}}
    \begin{bmatrix}
        1 & -i \\
        -i & 1
    \end{bmatrix}
    \begin{bmatrix}
        1 \\
        0
    \end{bmatrix}
    = \frac{1}{\sqrt{2}}
    \begin{bmatrix}
        1 \\
        -i
    \end{bmatrix}
    = \frac{\ket{0} -i\ket{1}}{\sqrt{2}}
\end{equation}
\begin{figure}[H]
    \centering
    \includegraphics[width=0.4\textwidth]{Rxgatequantum.png}
    \caption{\(\frac{\ket{0} -i\ket{1}}{\sqrt{2}}\)}
\end{figure}
Quantum \(R_x\) gate can be realized for polarization by using quarter-wave plate with fast axis at angle \(\theta\) with respect to the horizontal axis, where \(\theta = \frac{\pi}{4}\):
\begin{align}
	&\text{QWP}(\frac{\pi}{4})= 
	\begin{bmatrix}
		\cos^2 (\frac{\pi}{4}) + i \sin^2 (\frac{\pi}{4}) & (1-i) \sin(\frac{\pi}{4})\cos(\frac{\pi}{4})     \\
		(1-i) \sin(\frac{\pi}{4})\cos(\frac{\pi}{4})      & \sin^2 (\frac{\pi}{4}) + i \cos^2(\frac{\pi}{4})
	\end{bmatrix} \\[2ex]
	&= \begin{bmatrix}
		\frac{1}{2} + i \frac{1}{2} & (1-i) \frac{1}{2} \\
		(1-i) \frac{1}{2}          & \frac{1}{2} + i \frac{1}{2}
	\end{bmatrix} = 
	\frac{1}{2} 
	\begin{bmatrix}
		1+i & (1-i) \\
		(1-i) & 1+i
	\end{bmatrix} = \frac{1+i}{2}
	\begin{bmatrix}
		1 & \frac{1-i}{1+i}\\
		\frac{1-i}{1+i} & 1
	\end{bmatrix} \\[2ex]
	&= \frac{1+i}{2}
	\begin{bmatrix}
		1 & -i \\
		-i & 1
	\end{bmatrix}
\end{align}
We got the horizontal polarization \(\ket{0}_p\) going through QWP at \(\theta = \frac{\pi}{4}\) gives:
\begin{align}
    \text{QWP}(\frac{\pi}{4}) \ket{0}_p = \frac{1+i}{2}
    \begin{bmatrix}
        1 & -i \\
        -i & 1
    \end{bmatrix}
    \begin{bmatrix}
        1 \\
        0
    \end{bmatrix}
    = \underbrace{\frac{1+i}{2}
        \begin{bmatrix}
            1 \\
            -i
        \end{bmatrix}}_{\text{Jone's vector}}
    = \frac{1+i}{2} \ket{L}_p
\end{align}
We can extract the time-dependent electric field from the Jone's vector: \\ \\
\href{https://phys.libretexts.org/Bookshelves/Optics/BSc_Optics_(Konijnenberg_Adam_and_Urbach)/04%3A_Polarization/4.02%3A_Polarisation_States_and_Jones_Vectors}{LibreTexts}
\begin{align}
    \vec{E}(t) &= \text{Re} \left\{\ket{E} e^{-iwt} \right\} \\
    &\Rightarrow \vec{E_x}(t) = \text{Re}(E_x e^{-iwt}) \\
    &\Rightarrow \vec{E_y}(t) = \text{Re}(E_y e^{-iwt}) 
\end{align}
\begin{figure}[H]
    \centering
    \includegraphics[width=0.4\textwidth]{EfieldRx.pdf}
    \caption{\(\frac{\ket{0} -i\ket{1}}{\sqrt{2}}\) Time-dependent electric field}
\end{figure}
\subsection{Stokes vector measurement}
The Stokes vector component can be calulated from \(E_x\) and \(E_y\) of the Jones vector:
\begin{equation}
\text{Stokes} = \begin{bmatrix}
    I \\
    Q \\
    U \\
    V \\
\end{bmatrix}= 
\begin{bmatrix}
    |E_x|^2 + |E_y|^2 \\
    |E_x|^2 - |E_y|^2 \\
    2 \text{Re}(E_x E_y^*) \\
    2 \text{Im}(E_x E_y^*) \\
\end{bmatrix}
\end{equation}
\href{https://pol3he.sites.wm.edu/wp-content/uploads/sites/50/2020/01/measuring-Stokes-parameters.pdf}{Measuring the Stokes polarization parameters (Beth Schaefer)}
\end{document}