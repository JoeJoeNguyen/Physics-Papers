
\documentclass{article}
\usepackage[utf8]{inputenc}
\usepackage[english]{babel}
\usepackage[]{amsthm} %lets us use \begin{proof}
\usepackage[]{amssymb} %gives us the character \varnothing
\usepackage[]{setspace} %provides commands to set line spacing
\usepackage[left=0.75in, right=0.75in]{geometry}
\usepackage[colorlinks=true, linkcolor=blue, urlcolor=red, citecolor=blue]{hyperref}
\usepackage{xcolor}
\usepackage{soul}
\usepackage{amsmath}
\usepackage{amsfonts} % for \mathbb
\usepackage{amssymb}
\usepackage{graphicx}
\usepackage{float}
\usepackage{braket}





\begin{document}

\begin{center}
	\LARGE{Experiment}\\[1em]
	\large Son Nguyen\\[1em]
	%\large \today
\end{center}
\section{Polarization of Light}
\begin{itemize}
	\item \textbf{Light} is a wave composed of oscillating electric \(\vec{E}\) and magnetic \(\vec{B}\) field vectors. (Electromagnetic waves are transverse waves, meaning the oscillations are perpendicular to the direction of propagation.)
	\item \textbf{Polarization} refers to the direction of oscillations of light waves. Light typically vibrates in multiple directions. However, when light becomes polarized, it oscillates predominatly in a single direction.
\end{itemize}

\end{document}