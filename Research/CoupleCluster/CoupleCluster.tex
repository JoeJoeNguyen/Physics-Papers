\documentclass{article}
\usepackage[utf8]{inputenc}
\usepackage[english]{babel}
\usepackage[]{amsthm} %lets us use \begin{proof}
\usepackage[]{amssymb} %gives us the character \varnothing
\usepackage[]{setspace} %provides commands to set line spacing
\usepackage[left=0.75in, right=0.75in]{geometry}
\usepackage{hyperref}
\usepackage{xcolor}
\usepackage{soul}
\usepackage{amsmath}
\usepackage{amssymb}
\usepackage{graphicx}
\usepackage{float}

\begin{document}
%\maketitle %This command prints the title based on information entered above
\begin{center}
    \LARGE{Unitary Coupled Cluster}\\[1em]
    \large Son Nguyen\\[1em]
    %\large \today
\end{center}

\onehalfspacing
\begin{itemize}
    \item We need to parameterized the wavefunction in a way to efficiently explore the sector of the molecular Fock space that contains the desire solution.
    \item Designing quantum algorithms for quantum chemistry calculations requires reformulating the fermionic problem into qubit operators:
    \begin{enumerate}
        \item Mapping of the original electronic structure Hamiltonian into corresponding qubit Hamiltonian.
        \begin{itemize}
            \item The mapping will work in the second quantization formalism of quantum mechanics.
            \item The SQ Hamiltonian is formulated in the Hartree-Fock basis and mapped to the qubit space using either JW, BK, or parity mapping. 
        \end{itemize}
        \item The preparation suitable trial wavefunctions.
        \begin{itemize}
            \item The trial wavefunctions are constructed applying a set of perturbations ('excitations') to the HF ground state wavefunction, \(|\Phi_0\rangle\). \hl{ The pertubations are controlled by a set of parameters, (gate angles) that are then optimized until convergences is reached.}
            \item The trial wavefunction, can be prepare with either two main statergies. First, one can translate classical approaches (full Configuration Interaction, Coupled Cluster, etc.) in the qubit language. The second approache, named heuristic sampling, prepares the trial state using single qubit rotations and hardware efficient entangler blocks that span the whole qubit register. 
        \end{itemize}
        \item The development of an optimization scheme that converges to a ground state solution compatible with the nature of the quantum circuit.
    \end{enumerate}
\end{itemize}
\subsection*{The Coupled Cluster Theory}
Within the second quantization framework, the CC wavefunction is described as an expnentiated excitation operator acting upon a reference determinant, usually the Hartree-Fock state.
The Hartree-Fock state is the single-occupation vector with the lowest energy, it also is a close approximation to the ground state.6
\subsection*{The UCCSD Ansatz}
The UCC Ansatz can be written as:
\[\hat{U} = exp \left[\sum_{n}^{}\hat{T}_n - \hat{T}^\dagger_n\right]\] 
Where in the UCC the trial wavefunction is parameterized using the following Ansatz:
\[|\Psi(\vec{\theta}) \rangle = e^{\hat{T}(\vec{\theta}) - \hat{T}^\dagger (\vec{\theta})} |\Phi_0\rangle\]
Where:
\begin {itemize}
    \item  \(|\Psi(\vec{\theta}) \rangle\) is the trial wavefunction.
    \item \(|\Phi_0 \rangle = \prod_{i=1}^{N} \hat{a}^\dagger_i |vac\rangle\) is the vaccum state in the N-particles Fock space. (Ground state). It provides a good initial approximation to the true ground state of the molecular system. In this state, electrons occupy the lowest energy orbitals according
    to the Pauli exclusion principle, giving the a single Slater determinant. If an orbital is occupied, the corresponding qubit is set to \(|1\rangle\), if an orbital is unoccupied, the corresponding qubit is set to \( |0\rangle \).  The UCCSD ansatz then 
    build upon this initial state by adding excitations (through rotation and entangling gates) that captures electron correlations, which the HF state alone does not include. The UCCSD ansatz can adjust the electron configuration to better approximate the true correlated ground state of the molecule. 
    \item \(\hat{T}(\vec{\theta}) = \hat{T}_1 + \hat{T}_2 + \hat{T}_3 + \dots + \hat{T}_n\) is the excitation operator to order n, but we are only using single and double excitation.\\
    Therefore: 
    \[\hat{U}(\vec{\theta}) = e^{\hat{T}_1 (\vec{\theta})+ \hat{T}_2 (\vec{\theta}) - \hat{T}^\dagger_1 (\vec{\theta}) - \hat{T}^\dagger_2 (\vec{\theta})}\]
    \\
    The single excitation operator, which move one electron from an occupied orbital to a virtual orbital.
    \[\hat{T}_1(\vec{\theta}) = \sum_{i;m}^{}\hat{a}^\dagger_m \hat{a}_i\]
    The single excitation operator, which move two electrons from an occupied orbital to two virtual orbitals.
    \[\hat{T}_2(\vec{\theta}) = \sum_{i,j;m,n}^{}\hat{a}^\dagger_m \hat{a}^\dagger_n \hat{a}_i \hat{a}_j\]
    
\end{itemize}



\subsection*{The Heuristic Ansatz}


\end{document}