\documentclass{beamer}
%Information to be included in the title page:
\usepackage{braket}
\usecolortheme{beaver}
\usetheme{CambridgeUS}
\title{Quantum Machine Learning in Quantitative Finance}
\subtitle{A Review of Applications of Quantum Machine Learning for Quantitative
Finance}
\author{Son Nguyen}
\institute{Stevens Institute of Technology}
\date{2025}
\begin{document}
\logo{\includegraphics[width=1.5cm]{stevens.pdf}}
\frame{\titlepage}
\begin{frame}
	\frametitle{Introduction}
	"By leveraging the unique properties of quantum systems,
	such as superposition and entanglement, QML techniques have the potential to improve traditional
	methodologies in finance. These techniques promise to enhance predictive capabilities in areas such
	as portfolio optimization, market prediction, trading, pricing, and risk management"
\end{frame}
\begin{frame}
	\frametitle{Introduction}
	\begin{block}{Challenges}
		\begin{itemize}
			\item Classical Bottle Neck: scalability struggling when the dimension of the data increases.
			\item Monte Carlo Simulation: a method used to estimate the value of an option by simulating the underlying asset's price path. However, it can be computationally expensive and time-consuming, especially for high-dimensional problems.
		\end{itemize}
	\end{block}
	\begin{block}{Quantum Advantage}
		\begin{itemize}
			\item Handling High Complexity: Quantum algorithms can offer exponetial speedups.
			\item Solving Combinatorial Optimization Problems.
			\item Speeding up Monte Carlo Simulations: Quantum algorithms can potentially speed up the process of generating random samples, which is a key component of Monte Carlo simulations.
		\end{itemize}

	\end{block}
\end{frame}
\begin{frame}
	\frametitle{Portfolio Optimization}
	\begin{block}{Objective}
		Build an investment portfolio that maximizes returns while minimizing risk. To achive this, we need to choose carefully the option of assets such as
		stocks, bonds, and other financial instruments. The key is diversification, which means spreading investments across different assets that has \textbf{little to no correlation} to minimize risk and domino effect.
	\end{block}
\end{frame}

\begin{frame}
	\frametitle{Portfolio Optimization - Classical Approach}
    

\end{frame}


\end{document}